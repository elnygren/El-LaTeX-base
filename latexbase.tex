\documentclass[12pt, oneside]{article}


% You may add the packages you need here
\usepackage{amssymb,amsmath}
\usepackage[margin=1in]{geometry}
\usepackage{textpos}
\usepackage{wrapfig}
\usepackage{fancyhdr}


%Suppert for languages requireing UTF-8 and T1, such as Nordic
\usepackage[utf8]{inputenc}     
\usepackage[T1]{fontenc}
\usepackage[finnish]{babel} 

% Used to format code
% Pygments must be installed separately
% https://code.google.com/p/minted/
\usepackage{minted}

%Side by side columns
\usepackage{multicol}



% Begin Document
\begin{document}

% package:fancyhdr - Header with text and page numbers
\pagestyle{fancy}
\lhead{ Your Name }
\rhead{ \thepage}


\begin{center}
\textbf{\Large TITLE HERE}
\end{center}

\section*{I - Math examples}

\subsection*{1. Basic Proof} 

$(K_{12}, +)$ is a group simulating a 12-hourd clock. It contains the whole numbers of a from 1 to 12 and the operation $+$ allows adding times together. For example, $11+10 = 9$. Prove that $A = \{5, 10, 15\}$ is a subgroup of $(K_{15}, +)$. \\

Proof: 

\begin{center}
\begin{tabular}{ l | l l l }
 +   & 5   & 10 & 15 \\ \hline
 5   & 10 & 5   & 10 \\
 10 & 15 & 5   & 10 \\
 15 & 5   & 10 & 15 
\end{tabular}
\end{center}

\begin{center}
\begin{align*}
&g,h \in A \rightarrow gh \in A	\tag{Closure} \\
&15 * x = x * 15 = x, x \in K_{15} \text{ ja } 15 \in A \tag{Identity element of K in A} \\
&5^{-1} = 10, 10^{-1} = 5 \text{ ja } 15^{-1} = 15 \tag{Each element's $(\in A)$ inverse $\in A$}
\end{align*}
\end{center}

\subsection*{2. Matrices} 

$\begin{bmatrix}
1 & 1 & -1 \\
0 & -1 & 1 \\
 \end{bmatrix}
 *
 \begin{bmatrix}
1 & 1 \\
-1 & -1 \\
-1 & 0
 \end{bmatrix} \\ \\ \\
 =
  \begin{bmatrix}
  1*1 + 1*(-1) + (-1)*(-1) & 1*1 + 1*(-1) + (-1) * 0\\
  0*1 + (-1)*(-1) + 1*(-1) &  0*1 + (-1)*(-1) + 1*0\\
 \end{bmatrix} \\ \\ \\
 =
 \begin{bmatrix}
  1 & 0  \\
  0 & 1
 \end{bmatrix}$\\

\pagebreak

\subsection*{3. Multicolumns} 

Prove that the operation $\odot$ can't be defined in the set of rational numbers followingly:

$$\frac{m}{n} \odot \frac{k}{l} = \frac{m+k}{m^2+l^2}$$\\

\noindent Clearly  $\frac{1}{1} = \frac{2}{2}$, however:

\begin{multicols}{2}
\begin{align*}
\frac {m} {n} \odot \frac {k} {l} & = \frac{m+k}{m^2+l^2}\\
& = \frac{1+1}{1^2+1^2} \\
& = \frac{2}{2} 
\end{align*}

\columnbreak

\begin{align*}
\frac {m} {n} \odot \frac {k} {l} & = \frac{m+k}{m^2+l^2}\\
& = \frac{2+2}{2^2+2^2} \\
& = \frac{4}{8}
\end{align*}
\end{multicols}

\pagebreak
\section*{I - Code examples}

\subsection*{4. Python code} 

\begin{minted}{python}
from random import randint

def dice_probability_test(n):
	numbers = [0, 0, 0, 0, 0, 0]
	for i in range(1, n):
		number = randint(1,6)-1
		numbers[number] = numbers[number] + 1
	return numbers
\end{minted}


\end{document}